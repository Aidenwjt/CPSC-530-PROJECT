\documentclass[11pt]{article}

\usepackage{hyperref}

\setlength{\textwidth}{430pt}\setlength{\oddsidemargin}{11pt}

\title{CPSC 530 \\ \Large Group 17 Project Proposal}
\author{
Aiden Taylor - 30092686 - Computer Science
\and
Noah Pinel - 30159409 - Computer Science
\and
Ty Irving - 30105319 - Computer Science
}
\date{Feb. 5th, 2023}

\begin{document}

\maketitle
\newpage

\section*{Introduction to the Topic}
%Introduction to the topic and what you would like to study/do, and identifying the main paper(s)/reference(s) that you will be using.
The proposed project concerns the design and analysis of algorithms used in Data Compression.
When storing, transferring, sending, or receiving large files it is important to have efficient data compression and decompression algorithms.
However, this creates a problem, specifically, in the decision of which algorithm to use in different scenarios.
In order to analyze if a choice of algorithm was the most efficient, one has to implement this algorithm then compare it against the other options,
which is what we plan to do.
In our case, we plan to implement
the lossless data compression algorithm, Bit Code Complete Binary Tree (BCCBT), described in this paper. 
We will then compare this implementation against other sophisticated compression utilities,
such as Huffman encoding, the Lempel-Ziv family of algorithms, Dynamic Markov Compression, and Arithmetic Encoding.

\subsection*{Papers and References}
\begin{itemize}
  \item Paper: A study in compression algorithms\\
        Author: Mattias Håkansson Sjöstrand\\ 
        \href{http://bth.diva-portal.org/smash/record.jsf?faces-redirect=true&aq2=%5B%5B%5D%5D&af=%5B%5D&searchType=SIMPLE&language=en&pid=diva2%3A830266&aq=%5B%5B%5D%5D&sf=all&aqe=%5B%5D&sortOrder=author_sort_asc&onlyFullText=false&noOfRows=50&dswid=482}{Link to paper}
  
  \item Paper: Comparison Study of Lossless Data Compression Algorithms for Text Data \\
        Authors: A. Bhattacharjee, T. Bej, S. Agarwal\\ 
        \href{https://www.semanticscholar.org/paper/Comparison-Study-of-Lossless-Data-Compression-for-Bhattacharjee-Bej/ac777e46e7473c9e20ae94cceb58dcd4c058ce01}{Link to paper}
  
  \item Paper: Comparison of lossless data compression methods \\
        Authors: D. Berz, M. Engstler, M. Heindl, F. Waibel\\
        \href{https://www.researchgate.net/publication/335572104_Comparison_of_lossless_data_compression_methods}{Link to paper}

\end{itemize}

\newpage
% I might have autism - P(a) = .9, =)
% Outline of the proposed work and implementation/experiments for the project.
\section*{Outline of Proposed Work and Implementation}
What is important is how we will be implementing experiments for this project and how this will
help us understand which algorithms are most efficient.  The way that we will be doing this is by
using implementing the lossless data compression algorithm Bit Code Complete Binary Tree (BCCBT) 
along with other compression techniques and testing multiple factors when files are compressed and 
decompressed using these.  By being able to compare these factors that we will get by compressing and
decompressing files we will be able to evaluate when certain algorithms are used for any given scenario.
In order to be able to properly test and ensure that we have multiple different types of inputs to begin 
compressing and decompressing multiple different file types of sizes and content will be used in order to 
see if there is any effect on how certain files compare when using specific algorithms and if there is a case
where one will be used over another this should be evident when comparing the different types of 
factors against each other for the different types of algorithms.
\end{document}
