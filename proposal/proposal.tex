\documentclass[11pt]{article}

\setlength{\textwidth}{430pt}\setlength{\oddsidemargin}{11pt}

\title{CPSC 530 - Group 17 Project Proposal}
\author{
Aiden Taylor - 30092686 - Computer Science
\and
Noah Pinel - 30159409 - Computer Science
\and
Ty Irving - 30105319 - Computer Science
}
\date{Feb. 5th, 2023}

\begin{document}

\maketitle
\newpage

\section*{Introduction to the Topic}
%Introduction to the topic and what you would like to study/do, and identifying the main paper(s)/reference(s) that you will be using.
The proposed project concerns the design and analysis of algorithms used in Data Compression.
When storing, transferring, sending, or receiving large files it is important to have efficient data compression and decompression algorithms.
However, this creates a problem, specifically, in the decision of which algorithm to use in different scenarios.
In order to analyze if a choice of algorithm was the most efficient, one has to implement this algorithm then compare it against the other options,
which is what we plan to do.
In our case, we plan to implement
the lossless data compression algorithm, Bit Code Complete Binary Tree (BCCBT), described in this paper. 
We will then compare this implementation against other sophisticated compression utilities,
such as Huffman encoding, the Lempel-Ziv family of algorithms, Dynamic Markov Compression, and Arithmetic Encoding.

\subsection*{Papers and References}
\begin{itemize}
  \item A study in compression algorithms. \\
        Mattias Håkansson Sjöstrand 
  
\item Comparison Study of Lossless Data Compression Algorithms for Text Data. \\
   Arup Kumar Bhattacharjee, Tanumon Bej, Saheb Agarwal 
  
  \item Comparison of lossless data compression methods. \\
       Dominic Berz, Marco Engstler, Moritz Heindl, Florian Waibel  
\end{itemize}

\newpage

\section*{Outline of Proposed Work and Implementation}
Outline of the proposed work and implementation/experiments for the project.

\end{document}
