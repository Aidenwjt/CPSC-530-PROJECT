\documentclass[11pt]{article}

\usepackage{enumerate}
\usepackage{hyperref}

\setlength{\textwidth}{430pt}\setlength{\oddsidemargin}{11pt}

%\title{CPSC 530 - Information Theory and Security \\ \Large Group 17 - Project Proposal}
% god this looks so fucking gross VOMIT plz read up on common practices of how to format papers in TeX (YIKES!) -Noah
\title{CPSC 530 - Information Theory and Security \\ Group 17 Project Proposal}

% Please start to use common practices Retards - Noah

\newcommand{\Aiden}{Aiden Taylor - B.Sc. in Computer Science}
\newcommand{\Noah}{Noah Pinel - B.Sc. in Computer Science}
\newcommand{\Ty}{Ty Irving - B.Sc. in Computer Science}
\author{
      \begin{tabular}
            { l  }
            \Aiden \\ \Noah\\ \Ty\\ 
      \end{tabular}
% Aiden Taylor - B.Sc. in Computer Science - Group 17
% \and
% Noah Pinel - B.Sc. in Computer Science - Group 17
% \and
% Ty Irving - B.Sc. in Computer Science - Group 17
}
\date{Feb. 5th, 2023}

\begin{document}

\maketitle
\newpage

\section*{Introduction to the Topic}
%Introduction to the topic and what you would like to study/do, and identifying the main paper(s)/reference(s) that you will be using.
The proposed project concerns the implementation and analysis of algorithms used in Data Compression.
When storing, transferring, sending, or receiving large files it is important to have efficient data compression and decompression algorithms.
However, this creates a problem, specifically, in the decision of which algorithm to use in different scenarios.
In order to analyze if a choice of algorithm was the most efficient, one has to implement this algorithm, then compare it against the other options,
which is what we plan to do.
In our case, we plan to implement
the lossless data compression algorithm, Bit Code Complete Binary Tree (BCCBT), proposed in our first referenced paper. 
We will then compare this implementation against other sophisticated compression utilities,
such as Huffman encoding, the Lempel-Ziv family of algorithms, Dynamic Markov Compression, and Arithmetic Encoding.
We will mainly follow the first referenced paper, since we plan to implement the algorithm proposed in this paper,
however, we plan to use the other two papers as inspiration for potential tests to analyze our implementation of BCCBT.
In the end, the goals of this project are to gain insight into the different factors and tests used to analyze compression algorithms,
and to gain experience in implementing compression algorithms that are to be used in practical scenarios.

\subsection*{Papers and References}
\begin{enumerate}[(1)]
\item Paper: A study in compression algorithms \\
        Author: Mattias Håkansson Sjöstrand\\ 
        \href{http://bth.diva-portal.org/smash/record.jsf?faces-redirect=true&aq2=\%5B\%5B\%5D\%5D\&af=\%5B\%5D\&searchType=SIMPLE\&language=en\&pid=diva2\%3A830266\&aq=\%5B\%5B\%5D\%5D\&sf=all\&aqe=\%5B\%5D\&sortOrder=author\_sort\_asc\&onlyFullText=false\&noOfRows=50\&dswid=482}{Link to paper}
  
  \item Paper: Comparison Study of Lossless Data Compression Algorithms for Text Data \\
        Authors: A. Bhattacharjee, T. Bej, S. Agarwal\\ 
        \href{https://www.semanticscholar.org/paper/Comparison-Study-of-Lossless-Data-Compression-for-Bhattacharjee-Bej/ac777e46e7473c9e20ae94cceb58dcd4c058ce01}{Link to paper}
  
  \item Paper: Comparison of lossless data compression methods \\
        Authors: D. Berz, M. Engstler, M. Heindl, F. Waibel\\
        \href{https://www.researchgate.net/publication/335572104_Comparison_of_lossless_data_compression_methods}{Link to paper}
\end{enumerate}

\newpage
% I might have autism - P(a) = .9, =) (no shit - aiden)
% Outline of the proposed work and implementation/experiments for the project.
\section*{Outline of Proposed Work and Implementation}

\subsection*{Proposed Work}
The following is the work that needs to be done to complete this project:
      \begin{enumerate}[1.]
            \item An analysis of the BCCBT algorithm and its pseudocode.
            \item An implementation of the BCCBT algorithm. 
            \item A comparison of our implementation of BCCBT against other compression utilities.
            \item A compilation of the comparison results into a 7-10 page technical paper.
            \item The design of three quiz questions that are relevant to our paper.
            \item A presentation that summarizes the main results of our paper.
      \end{enumerate}

\subsection*{Implementation}
We will first focus on implementing the BCCBT algorithm, which will be written in a
group agreed upon programming language.
Once we have successfully implemented the algorithm, and have a way to compress and 
decompress files with this algorithm,
we will begin analyzing our implementation by testing it with multiple data compression factors.
We will then compare these results against other commonly used compression algorithms mentioned above.
Some of the factors we will be using to analyze and compare these algorithms are as follows:
      \begin{enumerate}
            \item Compression Time 
            \item Decompression Time
            \item Saving Percentage = $\frac{Original\ File\ Size\ -\ Compressed\ File\ Size}{Original\ File\ Size}$
            \item Compression Ratio = $\frac{Compressed\ File\ Size}{Original\ File\ Size}$
      \end{enumerate}
Using these factors we will be able to see in what scenarios certain algorithms should and should not be used,
and we will be able to tell if our choice of algorithm was the most efficient, in the given scenario.
During our comparison testing, we plan on using different size .txt files
as our sources of information, however, if time permits, we may incorporate other file types as well.
%Remove first section and include some specific factors and info

%The following is how we plan to implement the main tasks of this project:


%      \begin{enumerate}[(i)]
%            \item Create algorithm that functions to compress different files
%            \item Analyze different factors of the created algorithm while its being run
%            \item Create presentation with \LaTeX
%     \end{enumerate}

\end{document}
