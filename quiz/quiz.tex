\documentclass[11pt]{article}

\usepackage{enumerate}
\usepackage{hyperref}

\setlength{\textwidth}{430pt}\setlength{\oddsidemargin}{11pt}

\title{CPSC 530 - Information Theory and Security \\ Group 17 Project Quiz}

\newcommand{\Aiden}{Aiden Taylor - B.Sc. in Computer Science}
\newcommand{\Noah}{Noah Pinel - B.Sc. in Computer Science}
\newcommand{\Ty}{Ty Irving - B.Sc. in Computer Science}
\author{
      \begin{tabular}
            { l  }
            \Aiden \\ \Noah\\ \Ty\\ 
      \end{tabular}
}
\date{Mar. 26th, 2023}

\begin{document}

\maketitle
\newpage

\section*{Quiz Questions and Answers}
\begin{enumerate}
\item[\textbf{Q1:}]
What specific type of Binary Tree is used in the implementation of the BCCBT Data Compression Algorithm?
Describe at least one discussed property of this type of Binary Tree.

\item[\textbf{A1:}]
The specific type of Binary Tree used is a \textbf{Complete Binary Tree}.
The relevant properties of this type of Binary Tree that will be discussed during our presentation are as follows.
\begin{itemize}
\item In a Complete Binary Tree all levels, EXCEPT potentially the last level, are completely full.
\item In a Complete Binary Tree nodes are filled in from left to right at each level.
\item In a Complete Binary Tree the number of nodes at level $n$ is $2^n$.
\end{itemize}

% ...

\end{enumerate}

\end{document}
